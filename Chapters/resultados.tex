\chapter{Resultados}
\label{chap:result}
Importante sempre ter um parágrafo introdutório para explicar os resultados encontrados.

%--------- NEW SECTION ----------------------
\section{Robôs Subaquáticos}
\label{underwater_robots}


\subsection{Considerações na Modelagem}
Esta seção aborda os princiapis elementos e considerações para modelagens de ROVs.

Assim como quase todos robôs móveis podem estar posicionado em relação a uma referência, a posição de um robô subaquatico pode ser representando, de acordo com \cite{Antonelli}, diante a sua posição e orientação.
Diante a um Frame fixo de referência, é possivel obter a posição, usando técnicas de sensoriamentom, de um veículo submerso que comunalmente representando como vetor.
%vetor = (x,y,z)

Para representar a rotação dos veiculos diante ao mesmo frame pode usar o vetor
%n =(r,p,y)
que é a representação de roll, pitch  e yaw.

A seguinte tabela apresenta os movimentos dos veículos Subaquáticos em relação ao destes. Esta tabela esta de acordo que demostrado em \cite[Antonelli]. Estes movimentos, surge, sway e heave, também são usados navegações marinhas. 





%lembre de usar o livro de fossen
\subsection{Sensores}

A presença de sensores em um sistema pode permiti a obtenção de dados de vários. A médida dos sensores podem ser direcionados para a propria dinânica de um sistema, neste caso um ROV, e o ambinete no qual este realizar suas ações.
Assim como classifica \cite[Towards], os sensores de um ROV pode distinguindo em dois grupos: \it[playloads sensors]  e \it[navigations sensor]. Os \it[payloads sensor] são unidades de medidas destinados a coletar dados do ambiente, alguns exemplos destes sensores saão: sensores CTD, destinados a mensurar a condutividades, temperatura e profundidade, sensores ADCP-Acoustic Doppler Curent pPofiler- são usados para medir a velocidade das correntes e câmeras para obter dados visuais.

Os navigations sensors são implementados com o foco na navegação do veículo, logos dados sobre a posição, orientantçã e veocidade são mensurados. Os navigations sensors maiscomum são: sensores de pressão, (DVL) mede o deslocamento Doppler no sinal de entrada refletido no fundo do mar para obter os dados da velocidade linear e sensores de inertia. Câmeras tambem do pode ser usada para obter dados da posiação de veiculos, assim como foi demostrado por \cite{visual_serving}, no qual foi ultizado duas câmeras par realizar um acomphamento da posição de ROV.

\subsection{Controle}
\subsection{Modelos}
\section{Revisão bibliográfica}
\subsection{Rede de Citação}
\subsection{Principais autores}
\subsection{Modelos}
\section{Mapa COnceitual}
\lipsum[1]

%--------- NEW SECTION ----------------------
\section{Testes integrados}
\label{sec:testi}
\lipsum[1]







