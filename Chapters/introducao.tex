\chapter{Introdução}
\label{chap:intro}

A dinâmica da socideade humanana sempre passou por mundanças em diveros aspectos:trabalho, social, ecônomico, politico, culutural e diveros outros. Nos ultimos anos, a presença de sistema com um certo grau de autonomia adiquiriu presença nas em diversas atividades que humanaas. Estas presenças acabam gerando influência de como o trabalho é visto e também na economia de diversas socideades.

evido ao crescimento de aplicações de robótica, e também, por algumas necessidaes que estão aparecento tanto para o ambiente residencal quanto industrial, há necessidade haver pessoas capacitadas para atuar na pesuisa e desenvolvimentos destas aplicações.

Durante o mês de Janeiro de 2022 até julho do mesmo ano, várias ativiaddes fora executadas no intuito de criar um base tanto na parte teorica quanto na prática para a formação de pesquisadores na área de robótica e sistemas autônomos.  As atividades foram: aplicação e contextualização para de conceitos e de estaisticas, atividades com robôs manipula odres, elaboração de artigos e resumo estendidos, elaboração de linha de pesquisa voltada para robótica sub-aquática.


%--------- NEW SECTION ----------------------
\section{Objetivos}
\label{sec:obj}
Apresnsenta os resultados alcançados no prgrama de formação de pesquisadores na área de robótica e sistemas autônomos.
\label{sec:obj}

\subsection{Objetivos Específicos}
\label{ssec:objesp}
Os objetivos específicos são:
\begin{itemize}
      \item Estudo de estaisticas.
      \item Estudo de robótica sub-aquática.;
      \item Elaboração de resumo estendido;
      \item Elaboração de artigos;
      \item Atividades com robôs manipula odres;
      \item Elaboração de linha de pesquisa voltada para robótica sub-aquática.
    
  \end{itemize}



%--------- NEW SECTION ----------------------
\section{Justificativa}
\label{sec:justi}

As atividades que encolvem sistemas autônoms estão crescendo quanto ao âmbito industrial e residencial. Novos robôs e novas soluções estão sendo desenvolvidas para estes ambiente.




%--------- NEW SECTION ----------------------
\section{Organização do documento}
\label{section:organizacao}

Este documento apresenta $5$ capítulos e está estruturado da seguinte forma:

\begin{itemize}

  \item \textbf{Capítulo \ref{chap:intro} - Introdução}: Contextualiza o âmbito, no qual a pesquisa proposta está inserida. Apresenta, portanto, a definição do problema, objetivos e justificativas da pesquisa e como este \thetypeworkthree está estruturado;
  \item \textbf{Capítulo \ref{chap:fundteor} - Fundamentação Teórica}: XXX;
  \item \textbf{Capítulo \ref{chap:metod} - Materiais e Métodos}: XXX;
  \item \textbf{Capítulo \ref{chap:result} - Resultados}: XXX;
  \item \textbf{Capítulo \ref{chap:conc} - Conclusão}: Apresenta as conclusóes, contribuições e algumas sugestões de atividades de pesquisa a serem desenvolvidas no futuro.

\end{itemize}
