\chapter{Introdução}
\label{chap:intro}
A necessidade de realizar intervenção em ambientes submersos orirunda de desejos humanos resultou em diversas aplicações.  As intervenções possuem diversos objetivos, parte considerável são voltados para fins insdustriais, ao exemplo do ramos destinados a área petrolofifera. De acordo com \cite{Bogue1}, uma aplicação que começou a ser implementada e desenvolvida, primeiramente pelas marinnha Americana e Britânica, nas décadas de 50 e 60 do século XX  foi o uso dos veiculos submarinos remotamente teleoperados - ROV.  

Inciamlente, os principais objetivo destes veículos eram voltadas para operações milatares. Hoje, Grande grupos industrias que possuem produções petrolofiresa, em area subemersas, ultilizam rovs para este em suas operações princiaplmente para realizar ações voltados para manutenção e inspeção.

Para grande parte das ações, os rovs precisam de pelo menos um operador para executar as suas tarefas. Os comandos que são, comunalmente, gerados pelo operador através do uso de jotystick. A Teleoperação dos ROVS são custosas, devido a necessidade de profissional bem treinado e capacitado. Atualmente, alguns Rovs, devido os avanços das técnicas de automação e sensoriamento, já possuem capacidade de realizar alguma tarefas de forma autonomas. 

Tornar um ROV com capacidades de executar ações autônomas resultar em custo menores nas operações, pois não há a necessidade da presença de um profissional na tarefa que foi automatizada, ao exemplo no uso de manipuladores em para atuar na manutenção de tubulações. Uma outra vantagem que pode ser atruibuidos  para os rovs com uma operabilidade parcilamente autonoma é o aumento da quantidade e qualiadades das aplicações. 
 
%--------- NEW SECTION ----------------------
\section{Objetivos}
\label{sec:obj}


Este estudo da arte tem como seu objetivo discutir aplicações, estruturas, arquiteturas e estratégias de Veículo subaquático operado remotamente que possuem capacidades de executar autônomas.
\label{sec:obj}

%\subsection{Objetivos Específicos}
%\label{ssec:objesp}
%Os objetivos específicos deste projeto são:
%\begin{itemize}
%      \item Desenvolver habilidades de gestão de projetos.
%      \item Desenvolver algoritmos utilizando ROS;
%      \item Utilizar visão computacional;
%      \item Simular um robô no gazebo;
%  \end{itemize}
%
%\subsubsection*{Objetivos específicos principais}
%\label{sssec:obj-principais}
%ok vendo Aqui
%
%
%\begin{equation}
%  E=mc
%\end{equation}
%
%
%\begin{equation*}
%  m=\frac{E}{c}
%\end{equation*}
%
%\begin{equation}
%  m=E/c
%\end{equation}


%--------- NEW SECTION ----------------------
\section{Justificativa}
\label{sec:justi}

Vários veículos subaquáticos operado remotamente são depende exclusivamente de operadores humanos. Um estudo sobre a aplicação de algumas das tarefas pode indicar como alguns destes veiculos podem ser ganhar a capacidade de executar algumas tarefas de forma autonoma que pode reduzir o custo de operações e aumentar qualiadade das operações.




%--------- NEW SECTION ----------------------
\section{Organização do documento}
\label{section:organizacao}

Este documento apresenta $5$ capítulos e está estruturado da seguinte forma:

\begin{itemize}

  \item \textbf{Capítulo \ref{chap:intro} - Introdução}: Contextualiza o âmbito, no qual a pesquisa proposta está inserida. Apresenta, portanto, a definição do problema, objetivos e justificativas da pesquisa e como este \thetypeworkthree está estruturado;
  \item \textbf{Capítulo \ref{chap:fundteor} - Conceito do projeto}: Discuti a ambientalização, a situação atul dos rovs e o mercado de autuação;
  \item \textbf{Capítulo \ref{chap:metod} - Metodologia}:Aprsenta os metódos e materias que foram ultilizados para compor este estudo da arte;
  \item \textbf{Capítulo \ref{chap:result} - Estudo do estado da Arte}: demostra o resultado do estudo da arte que foi executado;
  \item \textbf{Capítulo \ref{chap:conc} - Conclusão}: Apresenta as conclusóes, contribuições e algumas sugestões de atividades de pesquisa a serem desenvolvidas no futuro.

\end{itemize}
