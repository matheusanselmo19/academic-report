\chapter{Ambiente de Desenvolvimento}

\label{chap:ambientalizacao}
%--------- NEW SECTION ----------------------

Este capítulo tem  como foco apresentadar os ambientes que os ROVs costumam ser ultilizados, o mercado de atuação e posteriomente a situação atual no desenvolvimento  destes veículos.

\section{Ambientalização}
Os ROVs comunalmente são desginados para atuar em ambiente marinhos onde há presença de unidades dedicadas a exploração petrolífera e gás. O alcance da profundidade varia bastante, alguns ROVs são projetados para atuar até 100 m, profundidade média, enquantos outros podem chegar até 1000 m abaixo do nível do mar. Outros ambientes, onde há presença de ROVs, são lagos e rios, nestes locais o uso são dedicados a fins ambientais e inspeções de embarcações, assim como informa \cite{CBS}.  


\section{Mercado de Atuação dos ROVs}


De acordo com \cite{Bogue1}, exploração petrolifera submarina oferece grandes riscos para operadores. A manunetação e inspeção de estruturas submarinas podem oferecer uma grande fonte de risco para os profissionais que as realizam. O uso de robôs nestes campos ainda é fraco, mas deve apresentar um cresicimento considervél nos próximos anos assim como arfirma \cite{yahoo_finaces}  que apaonta que o mercado de ROV, juntamente com de AUV, deve atingir um meracdo de U\$ 7.2 bilhoẽs em 2026. Além de retirar a presença humana de ambientes insalubres,  o uso de ROVs também tende a diminuir os custos das operações.

Usando os apontamentos de \cite{Bogue2}, outras aplicações dos ROVs podem ser voltados para fins militares, ambientais e de pesquisa. Um atuação que os rovs deve estar presente fortemente é o deepsea minning. Em produndas regiões submersas é há uma forte presença de metais que são raros em locais do planeta. O uso de ROVs são fortes candidatos a realizar as explorações de mineração neste locais. Segundo \cite{Bogue2}, Canadian Nautilus Minerals e American Neptune Minerals são companhias de mineração que estão dedicando investimento para este ramo.


\section{Situação Atual do desenvolvimento}

Nos últimos anos houve um grande desenvolvimento de robôs submarinos.  A variedade estrutural dos robôs aumentaram, vaŕias ações do rovs passaram a ser automatizada e houve o surgimento de de veiculos subaquaticos completamente autonômo. Um excelente exemplo o é o FLATFISH, que além de sua possuir capacidade de realizar tarefas sem nenhema inteferência humana, é capaz de ficar alocado no fundo mar por 6 meses de operação.
Para os ROVs, há uma grande quantidade de pesquisa  e desenvolvimento para tornar este veiculo cada vez mais autonomo.




%--------- NEW SECTION ----------------------


%---------------picture------------------------------------
% \begin{figure}
%     \centering
%     \subfigure[Figure A]{\label{fig:a}\includegraphics[width=60mm]{./lq}}
%     \subfigure[Figure B]{\label{fig:b}\includegraphics[width=60mm]{./lq}}
%     \subfigure[Figure C]{\label{fig:c}\includegraphics[width=\textwidth]{./lq}}
%     \caption{Three simple graphs}
%     \label{fig:three graphs}
% \end{figure}
%----------------------------------------------------------

% \begin{figure}
%     \centering
%     \begin{subfigure}[b]{0.3\textwidth}
%         \centering
%         \includegraphics[width=\textwidth]{./lq}
%         \caption{$y=x$}
%         \label{fig:y equals x}
%     \end{subfigure}
%     \hfill
%     \begin{subfigure}[b]{0.3\textwidth}
%         \centering
%         \includegraphics[width=\textwidth]{./lq}
%         \caption{$y=3sinx$}
%         \label{fig:three sin x}
%     \end{subfigure}
%     \hfill
%     \begin{subfigure}[b]{0.3\textwidth}
%         \centering
%         \includegraphics[width=\textwidth]{./lq}
%         \caption{$y=5/x$}
%         \label{fig:five over x}
%     \end{subfigure}
%        \caption{Three simple graphs}
%        \label{fig:three graphs}
% \end{figure}


% %--------- NEW SECTION ----------------------
% \section{Assunto 2}
% \label{sec:ass2}
% flkjasdlkfjasdlkfjs

% \begin{table}[h]
%     \begin{subtable}[h]{0.45\textwidth}
%         \centering
%         \begin{tabular}{l | l | l}
%         Day & Max Temp & Min Temp \\
%         \hline \hline
%         Mon & 20 & 13\\
%         Tue & 22 & 14\\
%         Wed & 23 & 12\\
%         Thurs & 25 & 13\\
%         Fri & 18 & 7\\
%         Sat & 15 & 13\\
%         Sun & 20 & 13
%        \end{tabular}
%        \caption{First Week}
%        \label{tab:week1}
%     \end{subtable}
%     \hfill
%     \begin{subtable}[h]{0.45\textwidth}
%         \centering
%         \begin{tabular}{l | l | l}
%         Day & Max Temp & Min Temp \\
%         \hline \hline
%         Mon & 17 & 11\\
%         Tue & 16 & 10\\
%         Wed & 14 & 8\\
%         Thurs & 12 & 5\\
%         Fri & 15 & 7\\
%         Sat & 16 & 12\\
%         Sun & 15 & 9
%         \end{tabular}
%         \caption{Second Week}
%         \label{tab:week2}
%      \end{subtable}
%      \caption{Max and min temps recorded in the first two weeks of July}
%      \label{tab:temps}
% \end{table}